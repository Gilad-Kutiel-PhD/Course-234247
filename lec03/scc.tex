בהינתן גרף מכוון, 
$G = (V, A)$
נגדיר את היחס הבא:
$\{(u,v) : u \rightsquigarrow v \land v \rightsquigarrow u\}$,
כלומר צמתים $u$ ו-$v$ ביחס אם קיים מסלול מ-$u$ ל-$v$ וקיים מסלול מ-$v$ ל-$u$.
נשים לב שזהו יחס שקילות ולכן הוא מגדיר מחלקות שקילות. 
למחלקות שקילות אלו נקרא הרכיבים האי פריקים של $G$.

\textbf{דוגמה:}
\begin{center}
\begin{tikzpicture}[every node/.style={default node}, ->, x=1.5cm, y=1.5cm]

% NODES
\foreach[count=\i] \x \y in {
	0/0,0/1
	,1/0,1/1
	,2/0,2/1
	,3/0,3/1
}{
	\node(\i) at(\x,\y) {\i};
}

% EDGES
\foreach \u \v in{%
	1/2,1/3%
	,2/4%
	,4/1,4/3,4/6%
	,5/7%
	,6/5%
}{
	\draw[] (\u) -- (\v);
}

\foreach \u \v in{%
	3/5%
	,5/3%
	,6/8%
	,8/6%
}{
	\draw[] (\u) to[bend right] (\v);
}

\draw[-, orange, dashed, very thick]
(1.south west)	to[out=135, in=225]
(2.north west)	to[out=45, in=135]
(4.north east)	to[out=-45,in=-45,looseness=.7]
(1.south west)
;

\end{tikzpicture}
\end{center}

