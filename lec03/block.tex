\begin{definition}[רכיב אי פריק]
תת גרף $H$ של $G$ יקרא רכיב אי פריק אם $H$ אינו מכיל צומת מפריד
\end{definition}

במילים אחרות, $H$ נשאר קשיר גם אחרי הסרת צומת כלשהו ממנו.

\begin{definition}[רכיב אי פריק מקסימלי]
רכיב אי פריק $H$ של $G$ הוא מקסימלי אם לא קיים רכיב אי פריק אחר של $G$ שמכיל ממש את $H$
\end{definition}

מעתה כשנדבר על רכיבים אי פריקים נתכוון אך ורק לרכיבים אי פריקים 
\textbf{מקסימליים}
.

\textbf{דוגמה:}
\begin{center}
\begin{tikzpicture}[every node/.style={default node}]
\foreach[count=\i] \x \y in{0/0, 1/1, 1/-1, 2/0, 3/0, 4/1, 4/-1, 5/1}{
	\node(\i) at(\x, \y) {\i}:
}
\foreach \u \v in {%
	1/2%
	,1/3%
	,2/4%	
	,3/4%	
	,4/5%	
	,5/6%	
	,5/7%	
	,6/7%	
	,6/8%	
}{
	\draw (\u) -- (\v);
}

\draw[dashed]
(1.west) to[bend left]
(2.north) to[bend left]
(4.east) to[bend left]
(3.south) to[bend left]
(1.west)
;

\draw[dashed]
(4.west) to[out=90, in=90]
(5.east) to[out=-90, in=-90]
(4.west) 
;

\draw[dashed]
(5.west) to[bend left]
(6.north) to[out=0,in=0]
(7.south) to[bend left]
(5.west) 
;

\draw[dashed]
(6.west) to[out=90, in=90]
(8.east) to[out=-90,in=-90]
(6.west) 
;

\end{tikzpicture}
\end{center}

\begin{claim}
\label{claim:disjoint}
לשני רכיבים אי פריקים
$H_1$
ו-%
$H_2$
צומת אחד משותף לכל היותר
\end{claim}
\begin{proof}
נניח בשלילה שישנם שני רכיבים כאלו שחולקים את הצמתים $u$ ו-$v$.
נשים לב שלאחר שהסרה של צומת כלשהו כל רכיב בנפרד נשאר קשיר.
מעבר לכך הרכיבים עדיין חולקים צומת משותף ולכן הרכיב שמתקבל מאיחוד שני הרכיבים גם הוא אי פריק.
סתירה למקסימליות.
\end{proof}

\begin{claim}
הרכיבים האי פריקים מהווים חלוקה של קשתות הגרף
\end{claim}
\begin{proof}
קל לראות שכל קשת מוכלת ברכיב אי פריק אחד לפחות ומטענה 
\ref{claim:disjoint}
נובע שגם לכל היותר

\end{proof}

\begin{claim}
עבור צומת הפרדה $u$ עם בן מפריד $v$, שאר צמתי הרכיב האי פריק שמכיל את $uv$ נמצאים ב-%
$T_v$
\end{claim}
\begin{proof}
נשים לב ש-$u$ מפריד את 
$T_v$
משאר הגרף
\end{proof}
נסמן ב-$S$ את קבוצת הבנים המפרידים אז
\begin{corollary}
צמתי הרכיב האי פריק שמכיל את $uv$ הוא 
$T_v \cup \{v\} \setminus \bigcup_{w \in S:w \ne v} T_w$
\end{corollary}

%%%%%%%%%%%%%%%%%%%%%%%

נעדכן את המימוש של DFS כך שבריצת האלגוריתם נחשב גם את הרכיבים האי פריקים 
(נרצה לשמור לכל צומת את מספר הרכיב אליו הוא שייך)
\begin{enumerate}
\item
אתחול:
$\ldots$
\colorbox{yellow}{
$S' \leftarrow (s)$,
$b \leftarrow 0$,
לכל צומת 
$v \in V$
$B(v) \leftarrow -1$
}
\item
כל עוד המחסנית לא ריקה
\begin{enumerate}
	\item
	$u \leftarrow S.top()$
	\item 
	אם קיימת קשת 
	$uv$
	שחוצה את $U$ 
	($u \in U$)
		\begin{enumerate}
		\item $\ldots$
		\item
		\colorbox{yellow}{$S'.push(v)$}
		\end{enumerate}
	\item
	אחרת 
	\begin{enumerate}
		\item $v \leftarrow S.pop()$
		\item $\beta(v) \leftarrow i$

		\item 
				\colorbox{yellow}{
אם $v$ בן מפריד של $u$
		}
			\begin{enumerate}
				\item \colorbox{yellow}{$w \leftarrow S'.pop()$}
				\item \colorbox{yellow}{
כל עוד 
				$w \ne v$
				}
					\begin{itemize}
						\item \colorbox{yellow}{$B(w) = b$}
						\item \colorbox{yellow}{$w \leftarrow S'.pop()$}
					\end{itemize}	
			\end{enumerate}			
			\item \colorbox{yellow}{$b \leftarrow b + 1$}
	\end{enumerate}
	\item
	$i \leftarrow i + 1$
	\end{enumerate}
\end{enumerate}

