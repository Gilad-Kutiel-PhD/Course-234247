נניח שנתון לנו גרף לא מכוון וקשיר,
$G = (V, E)$,
כך שהקשתות שלו צבועות בכחול, אדום ולבן.
נסמן ב-%
$B$,$R$
ו-$W$ את החלוקה של $E$ בהתאם.
בנוסף, נתונה לנו פונקציית משקל
$w:E \to \mathbb{R}$

נגדיר שני כללים:

\begin{itemize}
\item 
הכלל האדום: בחר מעגל שלא מכיל קשתות אדומות, 
מבין הקשתות הלבנות על המעגל בחר אחת עם משקל מקסימלי וצבע אותה באדום.
\item
הכלל הכחול: בחר חתך שלא נחצה על ידי קשתות כחולות, 
מבין הקשתות הלבנות שחוצות את החתך בחר אחת עם משקל מינימלי וצבע אותה בכחול.
\end{itemize}

נניח בנוסף שקיים עפ"מ, 
$T \subseteq E$, 
שמכיל את כל הקשתות הכחולות ולא מכיל קשתות אדומות, כלומר:
$B \subseteq T$
וגם
$T \cap R = \emptyset$.
נוכיח את שתי הטענות הבאות:
\begin{claim}
אם ניתן להפעיל את הכלל האדום אז קיים עפ"מ שמקיים את התנאים הנ"ל גם אחרי הפעלת הכלל האדום.
\end{claim}

\begin{proof}
נניח שהפעלנו את הכלל האדום וצבענו את הקשת $e$ באדום.
אם 
$e \notin T$
סיימנו.
אחרת, נסתכל על החתך שנוצר על ידי הסרה של $e$ מ-$T$.
מכיוון ש-$e$ קשת על מעגל אז קיימת קשת נוספת, $e'$ שחוצה את החתך (ולא שייכת ל-$T$).
מכיוון של-$e$ משקל מקסימלי, העץ 
$T \setminus \{e\} \cup \{e'\}$
הוא עפ"מ.
\end{proof}

\begin{claim}
אם ניתן להפעיל את הכלל הכחול אז קיים עפ"מ שמקיים את התנאים הנ"ל גם אחרי הפעלת הכלל הכחול.
\end{claim}

\begin{proof}
נניח שהפעלנו את הכלל הכחול על חתך $S$ וצבענו את הקשת $e$ בכחול.
אם 
$e \in T$
סיימנו.
אחרת, נוסיף את $e$ ל-$T$, אז $e$ נמצאת על מעגל שאינו מכיל קשתות אדומות ולכן את $S$
חוצה קשת לבנה, $e'$.
מכיוון של-$e$ משקל מינימלי בחתך אז
$T \setminus \{e'\} \cup \{e\}$
עפ"מ.
\end{proof}

הטענות הבאות מראות שניתן להפעיל את הכללים לפי הצורך:

\begin{claim}
ניתן להפעיל את הכלל הכחול כל עוד הקשתות הכחולות אינן עץ.
\end{claim}

\begin{proof}
נניח שהקשתות הכחולות לא מהוות עץ, אז ביחס לקשתות הכחולות קיימים ב-$G$ שני רכיבי קשירות (לפחות)
כל רכיב קשירות כזה מהווה חתך מתאים.
\end{proof}

\begin{claim}
ניתן להפעיל את הכלל האדום כל עוד הקשתות ה%
\textbf{לא}
אדומות אינן עץ.
\end{claim}

\begin{proof}
אם הקשתות הלא אדומות אינן עץ אז קיים לפחות מעגל אחד לא אדום.
\end{proof}

\begin{corollary}
ניתן להפעיל את הכלל הכחול והאדום בסדר כלשהו כדי לקבל עפ"מ.
\end{corollary}
