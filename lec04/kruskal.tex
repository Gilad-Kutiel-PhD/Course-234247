אלגוריתם קרוסקל מתחיל מיער ללא קשתות, בכל שלב באלגוריתם נמזג שני רכיבי קשירות באמצעות
הקשת הקלה ביותר שמחברת שני רכיבי קשירות.
פורמלית:
\begin{enumerate}
\item
אתחול:
$\mathcal{C} \leftarrow \{\{v\} : v \in V \}$, 
\item
כל עוד 
$T = (V, B)$
אינו קשיר הפעל את הכלל הכחול:
\begin{enumerate}
\item
תהי $e$ הקשת הקלה ביותר שמחברת שני רכיבי קשירות,
$C_i, C_j$
\item
עדכן
$B \leftarrow B \cup \{e\}$,
$\mathcal{C} \leftarrow \mathcal{C} \setminus \{C_i, C_j\} \cup \{C_i \cup C_j\}$

\end{enumerate}
\end{enumerate}

\textbf{דוגמה:}

\begin{center}
\begin{tikzpicture}[
every node/.style={default node}
,x=1.7cm
,y=1.7cm
]

\foreach[count=\i] \x \y in {
0/0,0/1,1/0,1/1,-1/-1,-1/2,2/-1,2/2}{
\node(\i) at (\x, \y) {\i};
}

\foreach \u \v \w in {
3/1/1%
,1/2/2%
,2/4/2%
,4/3/2%
,5/6/3%
,6/8/4%
,8/7/5%
,7/5/6%
,6/2/7%
,8/4/8%
,7/3/8%
,5/1/8%
}{
\draw (\u) -- (\v) node[label above]{\w};
}

\begin{scope}[xshift=8cm]
\foreach[count=\i] \x \y in {
0/0,0/1,1/0,1/1,-1/-1,-1/2,2/-1,2/2}{
\node(\i) at (\x, \y) {\i};
}

\foreach[count=\i] \u \v in {
1/3%
,4/2%
,4/3%
,5/6%
,6/8%
,8/7%
,6/2%
}{
\draw (\u) -- (\v) node[label above, blue]{\i};
}
\end{scope}

\end{tikzpicture}
\end{center}



\textbf{הערות:}
\begin{itemize}
\item
ניתן לממש את האלגוריתם באופן הבא:
\begin{itemize}
\item
מיין את הקשתות בסדר לא יורד של משקלן
\item
עבור כל קשת לפי הסדר, אם היא מחברת שני רכיבי קשירות הפעל עליה את הכלל הכחול ועדכן את $C$
\end{itemize}
\item
זמן הריצה של מימוש כזה הוא 
$O(|E| \log |E|)$
עבור המיון ובנוסף לכל קשת צריך לבדוק אם היא מחברת שני רכיבים שונים ואם כן לעדכן את מבנה הרכיבים.
כזכור מקורס מבני נתונים קיים מימוש פשוט שעושה זאת בזמן ממוצע של 
$O(\log |V|)$
ולכן הזמן הריצה הכולל הוא 
$O(|E| \log |E|) = O(|E| \log |V|)$.

\end{itemize}
