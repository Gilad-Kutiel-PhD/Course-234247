בחלק זה נראה אלגוריתם גנרי שפועל על פי הכלל האדום.

\begin{definition}[קשת כבדה]
בהינתן מעגל, $C$, ופונקצית משקל, 
$w: E \to \mathbb{R}$,
קשת $e$ במעגל $C$ תקרא כבדה אם לא קיימת קשת אחרת במעגל $C$, 
$e'$
שמקיימת 
$w(e') > w(e)$.
\end{definition}
\begin{enumerate}
\item
אתחול:
$R \leftarrow \emptyset$ 
(קשתות אדומות)
\item
כל עוד ב-%
$T = (V, E \setminus R)$
יש מעגל
\begin{enumerate}
\item
בחר מעגל לבן, $C$, וקשת כבדה, $e$, על המעגל
\item
$R \leftarrow R \cup \{e\}$
\end{enumerate}
\end{enumerate}

\textbf{הערה:}
במקום לצבוע את הקשת באדום ניתן פשוט למחוק אותה מהגרף
\begin{claim}
האלגוריתם מחזיר עץ
\end{claim}
\begin{proof}
חוסר מעגלים נובע מיידית מהגדרת האלגוריתם.
נניח בשלילה שאלגוריתם פוגע בקשירות, כלומר קיים חתך כך שהאלגוריתם צובע באדום את כל הקשתות 
שחוצות אותו, נסתכל על הקשת האחרונה שנצבעה באדום ועל המעגל הלבן שגרם לה להיבחר לפי אבחנה 
\ref{observation:cycle}
קיימת קשת לבנה נוספת שחוצה את החתך - סתירה.
\end{proof}

\begin{claim}
האלגוריתם מחזיר עץ פורש מינימלי
\end{claim}

\begin{proof}
נוכיח באינדוקציה שבכל שלב בריצת האלגוריתם אוסף הקשתות הכחולות מוכל בתוך עץ פורש מינימלי כלשהו:

בסיס: טריוויאלי באתחול

צעד: לפי ההנחה קיים עפ"מ שמכיל את $i$ הקשתות הכחולות הראשונות.
נניח שהקשת, $e$, שהוספנו בשלב ה-%
$i+1$
לא שייכת לעפ"מ הנ"ל (אחרת סיימנו).
נוסיף את $e$ לעפ"מ, סגרנו מעגל.
במעגל זה קיימת קשת,
$e' \neq e$,
שחוצה את $S$, החתך (הלבן) שגרם להוספת $e$.
לפי הגדרת האלגוריתם
$w(e) \leq w(e')$
ולכן ניתן להחליף בין הקשתות הנ"ל ולקבל עפ"מ שמכיל גם את $e$.
\end{proof}


