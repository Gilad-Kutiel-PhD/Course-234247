בהינתן גרף (מכוון או לא) עם $n$ צמתים נרצה להדפיס טבלה בגודל 
$n \times n$ 
שבכניסה ה-$ij$ שלה נמצא ערך מסלול קל ביותר מצומת 
$i$
לצומת 
$j$.

ניתן, כמובן, לעשות זאת על ידי $n$ הרצות של אלגוריתם בלמן פורד או דייקסטרה ולמצוא את התשובה 
בסיבוכיות זמן של 
$O(n^2 m)$
ו-
$O(n m \log n)$
בהתאמה.
נראה שאפשר גם יותר טוב.

בהינתן גרף שצמתיו ממוספרים מ-$1$ עד $n$ נגדיר את 
$d_{ij}^k$
להיות משקל מסלול קל ביותר מצומת $i$ לצומת $j$ שיכול לעבור רק בצמתי ביניים עם אינדקסים ב-%
$[k]$.

\textbf{דוגמה:}
\begin{center}
\begin{tikzpicture}[
	every node/.style={default node}
	,x=2cm
	,y=2cm
]
\foreach[count=\i] \x \y in {
	0/0
	,0/1, 0/-1
	,1/1, 1/-1
	,2/0, 2/1, 2/-1
	,1/0
}{
	\node(\i) at(\x,\y){\i};
}

\foreach \u \v \w in {
	1/2/1, 1/3/1, 1/9/12%
	,2/4/1, 2/9/10%
	,3/5/1, 3/9/9%
	,4/6/1, 4/7/1, 4/9/7%
	,5/8/1, 5/9/5%
	,6/9/1%
	,7/6/1%
	,8/6/1, 8/9/2%
}{
	\draw (\u) -- (\v) node[label inside]{\w};
}
\end{tikzpicture}
\end{center}

למה שווים הערכים הבאים
$d_{19}^0$, $d_{19}^1$, $d_{19}^2$, $d_{19}^6$, $d_{19}^7$, $d_{19}^9$?

\begin{observation}
משקל מסלול קל ביותר מצומת $i$ לצומת $j$ שווה ל-%
$d_{ij}^n$.
\end{observation}

\begin{observation}
$d_{ij}^0 = w(ij)$
\end{observation}

נניח עכשיו שלכל $i$, $j$ ו-$k$ אנחנו מקבעים מסלול קל ביותר עבור
$d_{ij}^k$.

\begin{observation}
אם הצומת $k$ לא שייך למסלול שמתאים ל-%
$d_{ij}^k$
אז
$d_{ij}^k = d_{ij}^{k - 1}$
\end{observation}

\begin{observation}
אם הצומת $k$ שייך למסלול שמתאים ל-%
$d_{ij}^k$
אז
$d_{ij}^k = d_{ik}^{k - 1} + d_{kj}^{k - 1}$.
\end{observation}

\begin{corollary}
$$
d_{ij}^{k} = 
\begin{cases}
w(ij)														& k = 0
\\
\min\{d_{ij}^{k - 1}, d_{ik}^{k - 1} + d_{kj}^{k - 1}\}		& k > 0
\end{cases}
$$
\end{corollary}

\textbf{חישוב:}
נגדיר 
$n + 1$ 
מטריצות בגודל 
$n \times n$, $A^0, \ldots, A^n$, 
כאשר 
$A_{ij}^k = d_{ij}^k$.
נמלא את ערכי המטריצות מ-%
$A^0$
ועד
$A^n$
כאשר נשים לב כי בשביל למלא את מטריצה 
$A^k$
אנו צריכים לדעת אך ורק את ערכי המטריצה 
$A^{k - 1}$.

\textbf{סיבוכיות:}
אנחנו מחשבים 
$n^3$
ערכים וחישוב של ערך בודד לוקח 
$O(1)$ 
פעולות.
