בהינתן נוסחת נסיגה, $f$, נסתכל על גרף החישוב שלה,
$G_f$
זהו גרף מכוון שבו כל צומת מתאימה למצב (ערך פרמטרים מסוים לנוסחה) וקיימת קשת ממצב
$s_i$
למצב 
$s_j$
אמ"מ לצורך חישוב מצב
$s_i$
יש צורך לחשב את מצב 
$s_j$.

למשל עבור הקלט הבא לבעיית האינטרוולים:

\begin{center}
\begin{tikzpicture}[very thick]
\draw[very thin, gray!30, dashed] (0,0) grid (10, 1);
\foreach \y \a \b \w \i in {
	1/0/2/3/2 ,1/3/10/3/5%	
	,0/0/1/2/1 ,0/2/4/1/3 ,0/5/8/1/4 ,0/9/10/2/6%
}{
	\draw (\a,\y) -- (\b,\y) 
	node[label above]{\w} 
	node[label below]{$a_{\i}$};
}
\end{tikzpicture}
\end{center}

ונוסחת הנסיגה 
$$
\alpha(i) = \max
\begin{cases}
w(i) + \alpha(p(i))
\\
0 + \alpha(i - 1)
\end{cases}
$$

גרף החישוב יראה כך (ניתן אף לשים משקלים מתאימים על הקשתות):
\begin{center}
\begin{tikzpicture}[x=15mm,->, every node/.style={default node}]
\foreach \i in {0,...,6}{
	\node(\i) at(\i, 0) {$a(\i)$};
}

\foreach[count=\i] \j in {0,...,5}{
	\draw (\i) to[bend left] node[label below]{0} (\j);
}

\foreach \u \v \w in {
	6/4/2,5/2/3,4/3/1,3/2/1,2/0/3,1/0/2%
}{
	\draw (\u) to[bend right] node[label above]{\w} (\v);
}

\end{tikzpicture}
\end{center}

מה נדרוש מגרף החישוב ?
\begin{enumerate}
\item
חסר מעגלים
\item
לא גדול מדי
\item
ניתן לחשב את הערכים של הבורות
\end{enumerate}

דוגמה נוספת, כיצד יראה גרף החישוב עבור נוסחת הנסיגה של מסלולים קלים ביותר והקלט הבא:
\begin{center}
\begin{tikzpicture}[
	every node/.style={default node}
	,->
	,x=2cm
	,y=2cm
]
\node(s) at(0,0) {s};
\foreach[count=\i] \x \y in {
	1/-1,1/0,2/0,1/1
}{
	\node(\i) at(\x,\y) {\i};
}

\foreach \u \v \w in {
	s/1/1,s/2/3%
	,1/2/1%
	,2/3/1%
	,3/4/1%
	,4/2/4-%
}{
	\draw (\u) -- (\v) node[label inside]{\w};
}
\end{tikzpicture}
\end{center}


\begin{center}
\begin{tikzpicture}[
	every node/.style={default node}
	,->
	,x=3cm
	,y=1cm
]

\foreach \k in {0,...,4}{
	\foreach[count=\y] \n in {s,1,2,3,4}{
		\node(\n\k) at(\k,-\y) {$\n$};
	}
}

\foreach[count=\i] \j in {0,...,3}{
	\foreach \u \v \w in {
		s/1/1,s/2/3%
		,1/2/1%
		,2/3/1%
		,3/4/1%
		,4/2/4-%
	}{
		\draw (\v\i) -- (\u\j) node[label inside]{\w};
	}
}
\end{tikzpicture}
\end{center}
