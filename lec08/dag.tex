כאשר הגרף חסר מעגלים ניתן למצוא את משקל המסלול הקל ביותר על ידי נוסחה רקורסיבית פשוטה.
\\
\textbf{דוגמה:}
נתון גרף חסר מעגלים.

\begin{center}
\begin{tikzpicture}[every node/.style={default node}, x=2cm]
\foreach[count=\i] \x \y in {
	0/0
	,1/1
	,1/-1
	,2/-1
	,2/1
	,3/0
}{
	\node(\i) at(\x, \y){\i};
}

\foreach \u \v \w in {
	1/2/1%
	,1/3/2%
	,2/3/3%
	,2/5/1%		
	,3/4/2%
	,4/6/3%
	,4/5/1%
	,5/6/2%	
}{
	\draw[->] (\u) to node[label inside]{$\w$} (\v);
}
\end{tikzpicture}
\end{center}

ונתון מיון טופולוגי שלו:
\begin{center}
\begin{tikzpicture}[every node/.style={default node}, x=2cm]

\foreach \i in {1,...,6}{
	\node(\i) at(\i, 0){\i};
}

\foreach \u \v \w in {
	1/2/1%
	,2/3/3%
	,3/4/2%
	,4/5/1%
	,5/6/2%	
}{
	\draw[->] (\u) to[bend left] node[label inside]{$\w$} (\v);
}

\foreach \u \v \w in {
	1/3/2%
	,2/5/1%		
	,4/6/3%
}{
	\draw[->] (\u) to[bend right] node[label inside]{$\w$} (\v);
}

\end{tikzpicture}
\end{center}

\begin{claim}
$$
\delta(j) = \min_{ij \in E} \delta(i) + w(ij)
$$
\end{claim}

\begin{proof}
באינדוקציה על $j$.
\end{proof}

נשים לב שאם מחשבים את הערך של 
$\delta$
לפי סדר המיון הטופולוגי אז סיבוכיות החישובי היא 
$O(|E| + |V|)$.

\textbf{הערה:}
ניתן גם לשחזר את המסלולים הקלים ביותר על ידי שמירת מצביע לאבא של כל צומת.
\\
\textbf{שאלה:}
מדוע אי אפשר להשתמש באותה טכניקה גם עבור גרפים שמכילים מעגלים?
