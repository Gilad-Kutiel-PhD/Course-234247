המשפט המרכזי על רשתות זרימה קובע ש:
\begin{theorem}
תהי $f$ פונקציית זרימה ברשת
$(G, s, t, c)$.
התנאים הבאים שקולים:
\begin{enumerate}
\item
$f$
היא זרימת מקסימום.
\item
לא קיים מסלול שיפור (ב-%
$G_f$).
\item
קיים 
\stcut{}
ברשת שהקיבול שלו שווה ל-%
$|f|$.
\end{enumerate}
\end{theorem}

\begin{proof}
$ $
\\
\bm{$1 \Rightarrow 2$}
\\
נניח בשלילה שקיים ונקבל סתירה לפי למה
\ref{lemma:improve}.
\\
\bm{$2 \Rightarrow 3$}
\\
נסמן ב-$S$ את קבוצת הצמתים הישיגים מ-$s$ ב-%
$G_f$.
מלמה
\ref{stcut}
נובע כי
$f(S) = |f|$
ולפי ההגדרה של רשת שיורית נובע כי ב-$G$ מתקיים:
$$
\begin{array}{ll}
\forall e \in \delta(S) & f(e) = c(e)
\\
\forall e \in \rho(S) & f(e) = 0
\end{array}
$$
\\
\bm{$3 \Rightarrow 1$}
\\
מיידי מטענה 
\ref{claim:upper}.
\end{proof}

