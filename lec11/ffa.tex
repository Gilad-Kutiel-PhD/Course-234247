אלגוריתם גנרי למציאת זרימת מקסימום:

\begin{enumerate}
\item
אתחול: מציבים 
$f(e) \leftarrow 0$
לכל
$e \in E$

\item
כל עוד יש מסלול שיפור ברשת השיורית
$(G_f, s, t, c_f)$
\begin{enumerate}
\item
מציבים ב-$f$ את הזרימה המשופרת לפי למת שיפור הזרימה
\end{enumerate}
\item
פולטים את $f$
\end{enumerate}

\begin{claim}
אם וכאשר האלגוריתם עוצר, פלט האלגוריתם הוא זרימת מקסימום.
\end{claim}

\begin{proof}
מיידית ממשפט
\ref{theorem:mincutmaxflow}.
\end{proof}

\textbf{סיבוכיות:}
בשביל לנתח את הסיבוכיות של האלגוריתם יש לדעת:
\begin{enumerate}
\item
\label{item:iterations}
לאחר כמה איטרציות האלגוריתם עוצר (אם בכלל)
\item
מה הסיבוכיות של כל איטרציה
\end{enumerate}

נתמקד בסעיף
\ref{item:iterations}.
אם כל הערכים הם שלמים אז ברור שמספר האיטרציות חסום על ידי ערך זרימת המקסימום. 
אם הערכים רציונליים אפשר להכפיל אותם כך שכל הערכים יהיו שלמים.
הדוגמה הבאה מראה שבמקרה הגרוע זהו חסם עליון הדוק.

\begin{center}
\begin{tikzpicture}[x=1.5cm]

\foreach[count=\i] \x in {0, 5cm, 10cm}{
	\begin{scope}[xshift=\x]
		\node(s\i) at(0,0) {s};
		\node(1\i) at(1,1) {};
		\node(2\i) at(1,-1) {};
		\node(t\i) at(2,0) {t};

		\begin{scope}[yshift=-4cm]
			\node(rs\i) at(0,0) {s};
			\node(r1\i) at(1,1) {};
			\node(r2\i) at(1,-1) {};
			\node(rt\i) at(2,0) {t};
		\end{scope}
	\end{scope}
	
}

%% #1

\foreach \u \v \f \c in {
	s1/11/0/k,s1/21/0/k%
	,11/21/0/1,11/t1/0/k%
	,21/t1/0/k%
}{
	\draw (\u) to[] node[label inside] {\f/\c} (\v);
}

\foreach \u \v \c in {
	rs1/r21/k%
	,r11/rt1/k%
}{
	\draw (\u) to[] node[label inside] {\c} (\v);
}

\foreach \u \v \c in {
	rs1/r11/k%
	,r11/r21/1%
	,r21/rt1/k%
}{
	\draw[blue, dotted] (\u) to[] node[label inside] {\c} (\v);
}

%% #2

\foreach \u \v \f \c in {
	s2/12/1/k,s2/22/0/k%
	,12/22/1/1,12/t2/0/k%
	,22/t2/1/k%
}{
	\draw (\u) to[] node[label inside] {\f/\c} (\v);
}

\foreach \u \v \c in {
	rs2/r12/k-1%
	,r12/rs2/1%
	,r22/rt2/k-1%
	,rt2/r22/1%
}{
	\draw (\u) to[bend right] node[label inside] {\c} (\v);
}

\foreach \u \v \c in {
	rs2/r22/k%
	,r22/r12/1%
	,r12/rt2/k%
}{
	\draw[blue, dotted] (\u) to[] node[label inside] {\c} (\v);
}


%% #3

\foreach \u \v \f \c in {
	s3/13/2/k,s3/23/1/k%
	,13/23/0/1,13/t3/1/k%
	,23/t3/2/k%
}{
	\draw (\u) to[] node[label inside] {\f/\c} (\v);
}

\foreach \u \v \c in {
	rs3/r23/k-1%
	,r13/rs3/2,r13/rt3/k-1%
	,r23/rs3/1%
	,rt3/r13/1,rt3/r23/2%
}{
	\draw (\u) to[bend right] node[label inside] {\c} (\v);
}

\draw[blue, dotted] (r13) to[] node[label inside] {1} (r23);

\foreach \u \v \c in {
	rs3/r13/k-2%
	,r23/rt3/k-2%
}{
	\draw[blue, dotted] (\u) to[bend right] node[label inside] {\c} (\v);
}

\node[draw=none] at(10, 0) {$\dots$};
\node[draw=none] at(10, -4) {$\dots$};

\end{tikzpicture}
\end{center}

\begin{corollary}
ברשת זרימה בה כל הקיבולים שלמים, קיימת זרימת מקסימום עם ערכים שלמים.
\end{corollary}
