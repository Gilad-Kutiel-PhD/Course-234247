\begin{definition}[\stcut]
\stcut{}
הוא תת קבוצה של צמתים שמכילה את $s$ ואינה מכילה את $t$.
\end{definition}

נרחיב את הסימונים 
$\delta$, $\rho$
ו-%
$f$
עבור קבוצת צמתים, כלומר
$$\delta(S) \defeq \{uv \in E : u \in S \land v \notin S\}$$
$$\rho(S) \defeq \{uv \in E : u \notin S \land v \in S\}$$
ו-%
$$f(S) \defeq \sum_{e \in \delta(S)} f(e) - \sum_{e \in \rho(S)} f(e)$$

ונשים לב שלכל 
$S \subseteq V$
מתקיים
\begin{observation}
$f(S) = \sum_{v \in S} f(v)$
\end{observation}

\begin{lemma}
\label{lemma:stcut}
לכל
\stcut, $S$,
מתקיים
$f(S) = |f|$.
\end{lemma}

\begin{proof}
נשים לב שלכל צומת, 
$v \in S$,
שאינו $s$ מתקיים 
$f(v) = 0$
ובנוסף, לפי ההגדרה, מתקיים ש-%
$f(s) = |f|$.
\end{proof}

בפרט מתקיים ש:%
$$
|f| = f(V \setminus \{t\}) = -f(t)
$$

נסמן
$$
c(S) \defeq \sum_{e \in \delta(S)} c(e)
$$

\begin{claim}
\label{claim:upper}
לכל 
\stcut, $S$,
מתקיים
$|f| \leq c(S)$
\end{claim}

\begin{proof}
לפי למה
\ref{lemma:stcut}
ולפי הגדרת פונקציית זרימה מתקיים ש-%
$|f| = f(S) \leq c(S)$.
\end{proof}

הטענה האחרונה מאפשרת לנו למצוא חסם עליון על זרימת המקסימום.
בהמשך נראה כי זהו חסם עליון הדוק.
