עבור תת קבוצה של צמתים
$U \subseteq V$
בגרף 
$G = (V, E)$
נסמן ב-%
$N(U)$
את קבוצת השכנים של $U$, כלומר:
$$
N(U) \defeq \{v : uv \in E, u \in U, v \notin U\}
$$
\begin{theorem}[משפט הול]
בגרף דו צדדי
$G = (L \cup R, E)$
שמקיים
$|L| = |R| = n$
קיים שידוך מושלם אם ורק אם לכל 
$U \subseteq L$
מתקיים ש-%
$|U| \leq N(U)$.
\end{theorem}

נוכיח את המשפט באמצעות טיעונים על זרימה.

בהינתן גרף דו צדדי 
$G = (L \cup R, E)$
רשת זרימה מתאימה
$N = (G', c)$
ו-%
\stcut{} $S$
נסמן ב-%
$U \defeq S \cap L$
וב-%
$W \defeq S \cap R$.

\begin{claim}
אם $S$ 
\stcut{}
מינימום אז לא קיים צומת 
$v \in W \setminus N(U)$.
\end{claim}

\begin{proof}
אם קיים אז 
$S \setminus \{v\}$
חתך קטן יותר.
\end{proof}

\begin{claim}
קיים
\stcut{}
מינימום כך ש-%
$W = N(U)$.
\end{claim}

\begin{proof}
נסתכל על
\stcut{}
מינימום,
$S$
שממזער את 
$|N(U) \setminus W|$
(השכנים של $U$ שלא ב-$S$)
נניח בשלילה שקיים צומת 
$v \in N(U) \setminus W$
אז 
$S \cup \{v\}$
חתך עם ערך לא גדול משל $S$ - סתירה.
\end{proof}

\begin{proof}[הוכחת משפט הול]
אם $S$
\stcut{}
מינימום כך ש-%
$W = N(U)$
ומתקיים 
$|W| \geq |U|$
אז ערך החתך הוא לפחות $n$.
\end{proof}

\textbf{דוגמה:}
בדוגמה הבאה החתך הכתום מקווקו הוא חתך מינימום אבל אינו מכיל את כל השכנים של $U$, 
אם מוסיפים את השכן של $U$ שמחוץ לחתך נקבל שוב חתך מינימום.
החתך המנוקד הכחול אינו חתך מינימום ומכיל צומת שאינו שכן של $U$, 
אם נוציא את הצומת מהחתך נקבל חתך מינימום.
\begin{center}
\begin{tikzpicture}[every node/.style={
	circle
	,draw
	,fill=black
	,inner sep=2pt
	,minimum size=0
}]


\node[fill=none, draw=none](s) at(-2, 3) {s};
\node[fill=none, draw=none](t) at(4, 3) {t};

\foreach \i in {1,...,5}{
	\node(l\i) at(0,\i) {};
	\node(r\i) at(2,\i) {};
	\draw (s) -- (l\i);
	\draw (r\i) -- (t);	
}

\foreach \u \v in {
	l1/r1,l1/r2%
	,l2/r2%
	,l3/r2,l3/r3,l3/r4%
	,l4/r5,l4/r4%
	,l5/r5%
}{
	\draw[] (\u) -- (\v);
}

\draw[dashed, orange,-]
(s.west) to[out=90, in=225] 
(l5.north west) to[out=45, in=135] 
(r5.north east) to[out=-45, in=45]
(l4.south east) to[out=225, in=270]
(s.west)
;

\coordinate[above=3pt of l2](al2);
\draw[dotted, blue,-]
(s.west) to[out=90, in=180]
(al2) to[out=0, in=180]
(r3.north) to[out=0, in=45]
(r1.south east) to[out=225, in=0]
(l1.south) to[out=180, in=-90]
(s.west)
;

\end{tikzpicture}
\end{center}
