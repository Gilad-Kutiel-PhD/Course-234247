רוצים לשמור קובץ טקסט על הדיסק בצורה חסכונית.
אפשרות אחת היא לקודד כל תו בטקסט במספר סיביות קבוע.
מספר הסיביות שנזדקק לכל תו הוא 
$\lceil \log |\Sigma| \rceil$.
אפשרות נוספת היא לקודד כל תו במספר סיביות שונה.
נשים לב שקידוד כזה יכול להיות חסכוני יותר כאשר יש שוני בין שכיחויות התווים בטקסט.

\textbf{דוגמה:}

עבור הא"ב
$\{A,B,C,D\}$
והמחרוזת הבאה:
\textenglish{AAABCD}
קידוד באורך קבוע יהיה באורך 
$6 \times 2 = 12$.

לעומת זאת, אם נבחר את הקידוד הבא
\textenglish{
\begin{tabular}{|l|l|}
\hline
A &  1
\\
B &  01
\\
C &  001
\\
D &  000
\\
\hline
\end{tabular}
}


אז אורך הקידוד יהיה 11 בלבד.

\begin{definition}[קוד בינרי]
בהינתן א"ב סופי
$\Sigma$
קידוד הוא פונקציה שממפה כל תו בא"ב למחרוזת בינרית
$c:\Sigma \to \{0,1\}*$
\end{definition}

\begin{definition}[הרחבה של קוד]
הרחבה של קוד היא פונקציה
$c:\Sigma^* \to \{0,1\}$
שמוגדרת להיות
$c(t_1 \ldots t_k) = c(t_1) \ldots c(t_k)$
\end{definition}


